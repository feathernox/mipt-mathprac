\documentclass{article}
\usepackage[utf8]{inputenc}
\usepackage[T1]{fontenc}
\usepackage[english, russian]{babel}
\usepackage{textcomp}
\usepackage{amssymb}
\usepackage{amsmath}
\usepackage{amsthm}
\usepackage{listings}
\usepackage{multicol}
\usepackage{float}
\usepackage{hyperref}

\setlength{\topmargin}{-1.2in}
\setlength{\textheight}{10.3in}
\setlength{\oddsidemargin}{-0.4in}
\setlength{\evensidemargin}{-0.4in}
\setlength{\textwidth}{7in}
\setlength{\parindent}{0ex}

\newtheorem*{thm}{Теорема}
\newtheorem*{sttm}{Утверждение}
	
\title{Об экзистенциальном монадическом законе нуля и единицы для графа $G(n, p)$}
\author{Ремизова Анастасия, 599}

\begin{document}
	
	\maketitle
	
	\section{Определения и известные результаты}
	
	\subsection{Модель случайных графов Эрдеша-Реньи}
	
	Пусть ${G}$ --- множество, содержащее все конечные неориентированные графы $G = \langle |G|, E\rangle$. Определим функцию $p= p(n)$, $p: \omega \mapsto [0, 1]$. Определим вероятностную модель  ${G}(n, p) = \langle {G}, {\sf P}_p \rangle$: для графа $G$ c числом ребер $\epsilon$ определим ${\sf P}({G}(n, p) = G) = p^{\epsilon}(1-p)^{{n \choose 2} - \epsilon}$.
	
	\subsection{Логика}
	
	Рассмотрим формулы {\bf логики первого порядка} и {\bf монадической логики второго порядка} над случайными графами: сигнатура $\sigma$ содержит символы равенства $=$ и смежности $\sim$. Формулы логики первого порядка построены из атомарных переменных $x$, $y$, $x_1$, ... и двух предикатных символов, упомянутых выше, логических связок $\neg$, $\land$, $\lor$, $\rightarrow$ и кванторов $\forall$, $\exists$ по переменным. Формулы монадической логики (второго порядка), помимо этого, могут содержать переменные, обозначающие множества, в форме $x \in X$, и кванторы по ним.
	
	Отдельно рассмотрим {\bf экзистенциальную монадическую логику второго порядка} --- подмножество логики второго порядка, содержащее формулы, в которых все кванторы по множествам являются кванторами существования и стоят снаружи любой другой части формулы; на кванторы первого порядка ограничений не накладывается.
	
	\subsection{Закон 0 и 1}
	
	Будем рассматривать только конечные формулы. Назовем {\bf кванторной глубиной $q(\varphi)$ формулы $\varphi$} число вложенных кванторов в самой длинной цепи вложенных кванторов в этой формуле.
	
	Нас интересуют асимптотические свойства вероятности выполнения свойства $\varphi$ для случайного графа $\mathfrak{G}(n, p)$ --- обозначим эту вероятность ${\sf P}_{n, p}(\varphi)$, а особенно существование предела этой вероятности при $n \rightarrow \infty$, который мы назовем {\bf асимптотической вероятностью $\varphi$} --- ${\sf P}_{p} (\varphi) = lim_{n \rightarrow \infty} {\sf P}_{n, p} (\varphi)$. Если этот предел существует для любой формулы логики $L$, то мы говорим, что выполняется {\bf закон сходимости} для $L$ и $p$. Если, к тому же, для каждой формулы асимптотическая вероятность равна $0$ или $1$, мы говорим, что выполняется {\bf закон $0$ и $1$}. Если асимптотическая вероятность равна $1$, мы говорим, что $\varphi$ выполняется {\bf асимптотически почти наверное (а.п.н.)}. В случае, когда для всех формул логики $L$, имеющих глубину не более чем $k$, асимптотическая вероятность равна $0$ или $1$, мы говорим, что выполняется {\bf $k$-закон $0$ и $1$}. Разумеется, закон $0$ и $1$ (для логики $L$) выполняется тогда и только тогда, когда для любого натурального $k$ выполняется $k$-закон $0$ и $1$.
	
	В частности, если $L$ --- логика первого порядка, то мы называем эти законы {\bf FO-законами} (из-за того, что в английском языке логика первого порядка называется first order logic); если $L$ --- монадическая логика второго порядка --- то {\bf MSO-законами} (monadic second order logic); если $L$ --- экзистенциальная монадическая логика второго порядка --- то {\bf EMSO-законами} (existential monadic second order logic).
	
	\subsection{Игра Эренфойхта}
	Рассмотрим игру Эренфойхта $\textrm{EHR}^{FO}(A, B, k)$ на графах $A$ и $B$ для логики первого порядка. Есть два игрока, Новатор и Консерватор, и фиксированное число раундов --- $k$.
	
	На $\nu$-ом раунде ($1 \leq \nu \leq k$) Новатор выбирает либо вершину $x_\nu$ в $A$, либо вершину $y_\nu$ в $B$. Консерватор тогда выбирает вершину в противоположном графе.
	
	В случае монадической логики второго порядка игроки могут так же выбирать множества. Аналогично, на $\nu$-ом раунде ($1 \leq \nu \leq k$) игры $\textrm{EHR}^{MSO}(A, B, k)$ Новатор выбирает граф $A$ или $B$. Допустим, он выбрал $A$; противоположный случай аналогичен. После этого он выбирает либо вершину $x_\nu$, либо подмножество $X_\nu$ множества вершин $V(A)$ графа $A$. Если была выбрана вершина, то Консерватор выбирает вершину в $B$, иначе --- подмножество множества вершин $V(B)$ графа $B$.
	
	Cлучай экзистенциальной монадической логики $\textrm{EHR}^{EMSO}(A, B, k)$ аналогичен случаю обычной монадической логики, но если Новатор выбрал на некотором раунде вершину, он должен и на всех следующих раундах выбирать вершины.
	
	В конце игры $\textrm{EHR}^{FO}(A, B, k)$, если $x_i$, $y_i$ --- вершины, выбранные соответственно в графах $A$ и $B$ на $i$-ом раунде ($i \in \{1, ..., k\}$), то Консерватор выигрывает тогда и только тогда, когда:
	
	\begin{enumerate}
		\item $\forall i, j \in \{1, ..., k\}: (x_i \sim x_j) \Leftrightarrow (y_i \sim y_j) \land (x_i = x_j) \Leftrightarrow (y_i = y_j);$
	\end{enumerate}
	
	В конце игры $\textrm{EHR}^{MSO}(A, B, k)$ и $\textrm{EHR}^{EMSO}(A, B, k)$,  если $R_{V}$ --- множество номеров раундов, когда Новатор выбрал вершину, $R_{S}$ --- множество номеров раундов, когда Новатор выбрал  множество, $x_i$, $y_i$ --- вершины, выбранные соответственно в графах $A$ и $B$ на $i$-ом раунде ($i \in R_{V}$),  $X_i$, $Y_i$ --- множества, выбранные соответственно в графах $A$ и $B$ на $i$-ом раунде ($i \in R_{S}$), то Консерватор выигрывает тогда и только тогда, когда:
	
	\begin{enumerate}
		\item $\forall i, j \in R_{V}: (x_i \sim x_j) \Leftrightarrow (y_i \sim y_j) \land (x_i = x_j) \Leftrightarrow (y_i = y_j);$
		\item $\forall i \in R_{S} \forall j \in R_{V}: (x_j \in X_i) \Leftrightarrow (y_j \in Y_i).$
	\end{enumerate}
	
	Известен следующий результат [2, 3, 4, 5]:
	
	\begin{thm}
		Пусть $k$ --- произвольное натуральное число, $L$ --- логика из FO, MSO или EMSO. Случайный граф ${G}(n, p)$ подчиняется L-k-закону 0 и 1 тогда и только тогда, когда Консерватор а.п.н. имеет выигрышную стратегию в игре Эренфойхта $\textrm{EHR}^{L}({G}(n, p), {G}(m, p), k)$ при $n, m \to \infty$. 
	\end{thm}
	
	\subsection{Типы вершин}
	Введем вспомогательные понятия для графов. Рассмотрим неориентированный граф $A = (V, E)$ и $X \subset V$ такое, что $|X| \geq 2$, $|\overline{X}| \geq 2$. Назовем вершину $v$ \textbf{$X$-доминирующей}, если $\forall x: x \in X \land x \neq v \to x \sim v$, и \textbf{$X$-изолированной}, если $\forall x: x \in X \land x \neq v \to x \nsim v$. Вершину, обладающую одним из этих свойств, назовем \textbf{$X$-особой}. Вершину, не являющуюся ни $X$-доминирующей, ни $X$-изолированной, назовем \textbf{$X$-общей}. Аналогичные определения, т.е. $\overline{X}$-доминирующая, $\overline{X}$-изолированная и $\overline{X}$-общая для вершины графа, мы можем применить и к дополнению $X$: $x \in \overline{X}$ эквивалентно $\neg(x \in X)$.
	
	Таким образом, при нетривиальном делении графа на множества (т.е. и $X$, и $\overline{X}$ содержат хотя бы по 2 вершины) мы можем определить \textbf{$X$-тип} вершины, задаваемый свойством относительно $X$; аналогично можем получить и ее \textbf{$\overline{X}$-тип}. {\bf $A$-тип} вершины определим как упорядоченную пару из $X$-типа и $\overline{X}$-типа вершин.
	
	\subsection{Известные свойства случайных графов при $p = const$}
	
	\begin{thm}
		\textbf{Глебский, Коган, Лиогонький, Таланов, Фагин.} При $p = const$ FO-закон $0$ и $1$ выполнен.
	\end{thm}	
	
	Эта теорема эквивалентна тому, что Консерватор а.п.н. имеет выигрышную стратегию в игре Эренфойхта $\textrm{EHR}^{FO}({G}(n, p), {G}(m, p), k)$ при $n, m \to \infty$. 
	
	В частности, из этого может быть получено следующее свойство. Назовем \textbf{k-extension property} конъюнкцию следующих утверждений по всем возможным $a, b \in \mathbb{Z_{+}}$ таким, что $a + b = k$ :
	
	$$\forall v_1 ... \forall v_k: \left( \left(\bigwedge_{i \neq j} x_i \neq x_j \right) \rightarrow \exists z: \left( \bigwedge_{i} (z \neq v_i) \land \bigwedge_{i \leq a} (z \sim v_i) \land \bigwedge_{i > a} (z \nsim v_i) \right) \right).$$
	
	\begin{thm}
		${\sf P}_{p} (\textrm{k-extension property}) = 1$.
	\end{thm}

	\section{EMSO-3-закон 0 и 1}
	
	Наша задача --- проверить выполнимость 3-закона 0 и 1 в случае экзистенциальной монадической логики.
	
	Для этого мы применим теорему, приведенную в пункте 1.4 и покажем, что а.п.н. у Консерватора есть выигрышная стратегия в $\textrm{EHR}^{EMSO}({G}(n, p), {G}(m, p), 3), n, m \to \infty$.
	
	Считаем, что на первом раунде Новатор выбрал граф $A := G(n,p)$, случай выбора графа $B := G(m, p)$ симметричен.
	
	Заметим, что случай выбора $(X_i, \overline{X_i})$ эквивалентен случаю выбора $(\overline{X_i}, X_i)$: достаточно заменить условия вида $x_j \in X_i$ на $x_j \notin X_i$).
	
	Будем рассматривать игру Эренфойхта в том случае, когда на первом раунде Новатор выбрал множество. В ином случае во всех раундах Новатор будет выбирать только вершины, а не множества, и можно применять стратегию, аналогичную стратегии при игре Эренфойхта для логики первого порядка для того же графа, которая а.п.н. есть.
	
	Если на первом раунде Новатор выбирает множество всех вершин некоторого графа или пустое множество, то Консерватор может выбрать множество всех вершин другого графа или пустое множество соответственно. Далее условие $(x_j \in X_1) \Leftrightarrow (y_j \in Y_1).$ будет выполняться для любого выбора вершин графов $A$ и $B$, поэтому на следующем раунде можно выбирать такую стратегию, как была бы выбрана, если бы этого раунда не было.
	
	Если на первом раунде Новатор выбирает множество, состоящее из одной вершины, то можно действовать так, как будто была выбрана вершина, лежащая в этом множестве, и взять множество, соответствующее такой вершине. Действительно, если $X_1 = \{w\}$, то условие $x_j \in X_1$ эквивалентно $x_j = w$. Если на последующих раундах Новатор будет выбирать вершины, то можно применять стратегию для FO-3-логики, поскольку условие вершины $x_j = w$ не сильнее условия для выбора вершины: а.п.н. такая стратегия есть. Если же на втором раунде будет выбрано непустое множество, не содержащее все вершины графа (иначе тривиально), то достаточно проверить, лежит ли вершина, выбранная на первом раунде, в множестве, выбранном на втором. Если да, то проверить, есть ли еще вершины в этом множестве и выбрать в противоположном графе множество, содержащее вершину, лежащую в множестве, выбранную на первом раунде, из двух или одной вершин соответственно; аналогично в противоположном случае. Случай, когда множество содержит все вершины, кроме одной, также симметричен.
	
	Теперь будем считать, что $|X_1| \geq 2$ и $|\overline{X_1}| \geq 2$
	
	Заметим, что если у нас есть стратегия для случая, когда на втором раунде Новатор выбирает вершину, то у нас есть стратегия и в том случае, когда на втором раунде Новатор выбирает множество. Если Новатором, б.о.о., было выбрано множество в $A$, то достаточно проверить на пустоту/непустоту $X_1 \cap X_2$, $\overline{X_1} \cap X_2$ $X_1 \overline{X_2}$ и $\overline{X_1} \cap \overline{X_2}$, и выбрать $Y_2$, чтобы соответствующие свойства для $Y_1$ и $Y_2$ выполнялись. Очевидно, у Консерватора есть выигрышная стратегия на третьем раунде.
	
	Поэтому обозначим $X : = X_1$ и $Y := Y_1$.
	
	Довольно очевидно, что:
	
	\begin{sttm}
		Консерватор имеет выигрышную стратегию в игре Эренфойхта тогда, когда:
		\begin{enumerate}
			\item в $Y$ содержатся вершины тех и только тех $B$-типов, что в $X$ --- $A$-типов;
			\item в $\overline{Y}$ содержатся вершины тех и только тех $B$-типов, что в $\overline{X}$ --- $A$-типов;
		\end{enumerate}
	\end{sttm}
	
	
	\section{Свойства случайных графов при $p = const$}
	
	Очевидно следующее свойство:
	
	\begin{sttm}
		Пусть $G'$ --- произвольный неориентированный граф. Тогда а.п.н. в $G(n,p)$ существует индуцированный подграф, изоморфный $G'$.
	\end{sttm}
	
	\subsection{Свойства разбиения графа на $X$ и $\overline{X}$}
	
	Пусть $X$ --- подмножество множества вершин графа $G$ такое, что $|X| > 2$, $|\overline{X}| > 2$. Исследуем типы вершин, которые могут быть в $X$ и $\overline{X}$.
	
	\subsubsection{Случай, когда $X$ содержит $X$-доминирующую вершину}
	
	Рассмотрим случай, когда Новатор выбирает множество $X$, содержащее $X$-доминирующую вершину $x$. Заметим, что он симметричен случаям, если в $X$ была выбрана $X$-изолированная вершина или в $\overline{X}$ была выбрана $\overline{X}$-доминирующая/изолированная вершина.
	
	\begin{sttm}
		$X$ не содержит $X$-изолированной вершины.
	\end{sttm}
	
	\begin{sttm}
		А.п.н. $x$ не может быть $\overline{X}$-доминирующей.
	\end{sttm}
	
	\begin{proof}
		Из 1-extension property следует, что должна быть вершина, не соседствующая с $x$. 
	\end{proof}
	
	\begin{sttm}
		А.п.н нет $\overline{X}$-особой вершины, кроме, возможно, $x$.
	\end{sttm}
	
	\begin{proof}
		Пусть $x'$ --- $\overline{X}$-особая вершина, б.о.о. $\overline{X}$-доминирующая. Тогда для произвольной вершины $z$, не совпадающей с $x$ и $x'$, не может быть выполнено $x \nsim z \land x' \nsim z$: если $z \in X$, то $z \sim x$, а если $z \notin X$, то $z \sim x'$. Это противоречит 2-extension property, которое а.п.н. выполняется.
	\end{proof}
	
	\begin{sttm}
		В $\overline{X}$ а.п.н. не может быть только $X$-доминирующих или только $X$-изолированных вершин.
	\end{sttm}
	
	\begin{proof}
		Предположим противное: все вершины в $\overline{X}$, б.о.о., $X$-доминирующие. Тогда любая вершина из $X$ является $\overline{X}$-доминирующей, а это а.п.н. невозможно.
	\end{proof}
	
	
	\begin{sttm}
		В $\overline{X}$ а.п.н. найдется $X$-общая вершина.
	\end{sttm}
	
	\begin{proof}
		Предположим противное: рассмотрим вершину $x_d$, лежащую в $\overline{X}$, такую, что $x_d$ --- $X$-доминирующая, $x_i$ --- $X$-изолированная, и произвольные вершины $x'$ и $x''$. Тогда не существует вершины $z$ такой, что $x' \sim z \land x'' \nsim z \land x_d \nsim z$. Действительно, из условия $x' \sim z \land x'' \nsim z$ следует, что вершина $z$ $X$-общая, поэтому она лежит в $X$; но тогда она не может быть не соединена с $x_d$. Получили противоречие с 3-extension property.
	\end{proof}		
	
	\begin{sttm}
		Если в $X$ $x$ --- $\overline{X}$-изолированная вершина, то все вершины, кроме $x$, являются $X$-общими. 
	\end{sttm}
	
	\begin{proof}
		Рассмотрим произвольную вершину $x'$, не совпадающую с $x$. Из а.п.н. выполнения 2-extension property есть вершины $x_1$ и $x_2$, соединенные с $x$, такие, что $x_1$ соединена с $x'$, $x_2$ не соединена с $x'$. Из соединенности с $x$ следует, что $x_1$ и $x_2$ в $x'$; значит, $x'$ --- $X$-общая.
	\end{proof}
	
	
	
	\subsection{Случай, когда $X$ содержит только $X$-общие вершины}
	
	Рассмотрим случай, когда все вершины в $X$ --- $X$-общие и все вершины в $\overline{X}$ --- $\overline{X}$-общие.
	
	\begin{sttm}
		Не может существовать $\overline{X}$-особой вершины в $X$ и $X$-особой вершины в $\overline{X}$ одновременно. 
	\end{sttm}
	
	\begin{proof}
		Предположим противное. Пусть $x$ --- $\overline{X}$-особая вершина в $X$, $x'$ --- $X$-особая вершина в $\overline{X}$, б.о.о. доминирующие. Из 2-extension property следует, что а.п.н. существует вершина $z$, отличная от $x$ и $x'$ и не соединенная с ними. Значит, $z$ не может лежать в $X$, иначе она соединена с $x$, и в $\overline{X}$, иначе она соединена с $x'$.
	\end{proof}
	
	Таким образом, можно считать, что в $\overline{X}$ нет $X$-особых вершин.
	
	\begin{sttm}
		В $X$ а.п.н. есть $\overline{X}$-общие вершины.
	\end{sttm}
	
	\begin{proof}
		Предположим противное. Рассмотрим $x$, $x'$ --- произвольные различные вершины в $\overline{X}$ и $x_i$ --- произвольную $\overline{X}$-особую, б.о.о. $\overline{X}$-изолированную вершину в $X$. Из 3-extension property следует, что а.п.н. есть вершина $z$, соединенная с $x$ и с $x_i$ и не соединенная с $x'$ и с $x_d$. Тогда $z$, с одной стороны, должна лежать в $\overline{X}$, поскольку она не $\overline{X}$-особая; но
		тогда она не соединена с $x$. Противоречие.
	\end{proof}
	
	\subsection{Существование множества $Y$, обладающего определенными свойствами}
	
	\begin{sttm}
		Подграф, индуцированный множеством $Y$, содержащим произвольную вершину $y$ и всех ее соседей, кроме некоторого $y_1$, а.п.н. не является кликой, причем все вершины вне $Y$ $Y$-общие.
	\end{sttm}
	
	\begin{proof}
		Рассмотрим вершину $y'$, лежащую в $Y$. Из а.п.н. выполнения  3-extension property следует, что а.п.н. есть вершина $z$, соединенная с $y$ (т.е. лежащая в $Y$), не совпадающая с $y_1$ и не соединенная с $y'$, поэтому $y'$ --- $Y$-общая. Рассмотрим произвольную вершину $y'$, вне $y$. Для нее а.п.н. существует $z$, соседствующий с $y$, не совпадающий с $y_1$ (т.е. лежащий в $Y$) и не соседствующий с $y'$, а значит, $y'$ --- не $Y$-доминирующая. Аналогично она не $Y$-изолированная.
	\end{proof}
	
	\begin{sttm}
		Подграф, индуцированный множеством $Y$, содержащим всех соседей произвольной вершины $y$, кроме некоторого $y_1$, а.п.н. содержит только $Y$-общие, причем все вершины, кроме $y$, $Y$-общие.
	\end{sttm}
	
	\begin{proof}
		Рассмотрим произвольного соседа $y'$ вершины $y$. Из а.п.н. выполнения 3-extension property следует, что есть вершина $z$, отличная от $y_1$, соседствующая с $y$ (т.е. лежащая в $Y$) и соседствующая (несоседствующая) с $y'$, а потому $y'$ --- $Y$-общая. Для произвольной вершины $y'$, отличной от $y$ вне $Y$ существует $z$, соседствующая с $y$ (т.е. в $Y$), соседствующая(несоседствующая) с $y'$, а потому $y'$ --- $y$-общая.
	\end{proof}	
	
	
	\begin{sttm}
		Пусть $y$, $y_d$ --- соседствующие вершины. Множество $Y$, содержащее всех их общих соседей и саму $y$, а.п.н. содержит только $Y$-общие вершины, кроме $y$; все вершины в $\overline{Y}$ --- $Y$-общие либо $Y$-доминирующие.
	\end{sttm}
	
	\begin{proof}
		Из а.п.н. выполнения 3-extension property следует, что для произвольного $u$ из $Y$ есть вершина $z$, соединенная с $y$ и $y_d$, но не с $u$, а потому $u$ --- $Y$-общая. Если предположить, что существует $Y$-изолированная вершина $y_i$, то не существует $z$,  соединенной с $y$, $y_d$ и с $y_i$, что противоречит 3-extension property.
	\end{proof}	
	
	Аналогично:
	\begin{sttm}
		Пусть $y$, $y_i$ --- несоседствующие вершины. Множество $Y$, содержащее $y$ и пересечение cоседей $y$ и несоседей $y_i$ (и только их), а.п.н. содержит только $Y$-общие вершины, кроме $y$; все вершины в $\overline{Y}$ --- $Y$-общие либо $Y$-доминирующие.
	\end{sttm}
	
	\section{Стратегии для некоторых случаев}
	
	Построим выигрышную стратегию для $\textrm{EHR}(G(n, p), G(m, p), k)$, используя свойства, полученные в 3 пункте.

	Снова начнем с случая, когда Новатор выбирает множество $X$, содержащее $X$-доминирующую вершину $x$.
	
	\begin{sttm}
		Консерватор а.п.н. имеет стратегию, если в $X$ $x$ --- $\overline{X}$-изолированная вершина.
	\end{sttm}
	
	\begin{proof}
		В качестве $Y$ возьмем произвольную вершину $y$ в $B$ и всех ее соседей. 
	\end{proof}
	
	\begin{sttm}
		Консерватор а.п.н. имеет стратегию, если в $X$ нет $\overline{X}$-изолированной вершины и $\overline{X}$ содержит вершины всех трех типов относительно $X$.
	\end{sttm}
	
	\begin{proof}
		Если индуцированный подграф на $X$ является кликой, найдем в графе $B$  треугольник, а.п.н существующий, и возьмем его вершины в качестве множества. Иначе возьмем 3 вершины, одна из которых соединена ребрами с двумя другими, а две другие --- нет, а.п.н. такие существуют.
	\end{proof}
	
	\begin{sttm}
		Консерватор а.п.н. имеет стратегию, если в $X$ нет $\overline{X}$-изолированной вершины, $X$ --- не клика, и $\overline{X}$ содержит вершины только общего $X$-типа и одного из $X$-особых.
	\end{sttm}
	
	\begin{proof}
		Б.о.о., пусть $X$-особый тип в $\overline{X}$ --- $X$-доминирующий, для $X$-изолированного аналогично. Рассмотрим произвольные соседствующие вершины $y$ и $y'$, добавим в $Y$ $y$ и общих соседей $y$ и $y'$.
	\end{proof}
	
	\begin{sttm}
		Консерватор а.п.н. имеет стратегию, если в $X$ нет $\overline{X}$-изолированной вершины, $X$ --- не клика, и $\overline{X}$ содержит вершины только общего $X$-типа.
	\end{sttm}
	
	\begin{proof}
		Добавим вершину $x$ и всех ее соседей, кроме одного, в множество $X$.
	\end{proof}
	
	Рассмотрим случай, когда все вершины в $X$ --- $X$-общие и все вершины в $\overline{X}$ --- $\overline{X}$-общие.	
	
	\begin{sttm}
		Консерватор а.п.н. имеет стратегию, если в $X$ есть вершины всех $\overline{X}$-типов.
	\end{sttm}
	
	\begin{proof}
		Найдем 4 вершины $a$, $b$, $c$, $d$ такие, что ребра проведены между $a$ и $b$ и между $c$ и $d$; между остальными парами вершин ребра не проведены. Такие а.п.н. найдутся. Добавим их в множество $Y$. Из 4-extension property следует, что в $\overline{Y}$ найдутся вершины всех $Y$-типов.
	\end{proof}
	
	\begin{sttm}
		Консерватор а.п.н. имеет стратегию, если в $X$ есть вершины только одного из $\overline{X}$-особых и $\overline{X}$-общего типа.
	\end{sttm}
	
	\begin{proof}
		Рассмотрим случай, когда вершина $\overline{X}$-доминирующая. Возьмем произвольную вершину $y$ и добавим в $y$ всех ее соседей, кроме одного.
	\end{proof}
	
	\begin{sttm}
		Консерватор а.п.н. имеет стратегию, если в $X$ есть вершины только $\overline{X}$-общего типа.
	\end{sttm}
	
	\begin{proof}
		Возьмем вершину $y$ и добавим ее всех соседей, кроме одного, и одну вершину $y'$, с ней не соединенную.
	\end{proof}
	
	\section{Вывод}
	
	Чтобы решить задачу, нам требуется изучать свойство а.п.н. существования клик (антиклик) в случайном графе $G(n, p)$ при $p = const$ и их отношения к остальным вершинам, поскольку именно оно не покрывается логикой первого порядка.
	
	\section{Список литературы}
	
	[1] Ю.В. Глебский, Д.И. Коган, М.И. Лиогонький, В.А.Таланов, Объем и доля выполнимости формул узкого исчисления предикатов, \textit{Кибернетика}, 1969, \textbf{2}: 17-26.
	
	[2] P. Heinig, T. Muller, M. Noy, A. Taraz, Logical limit laws for minor-closed classes of
	graphs, to appear in J. Comb. Th. Ser. B., submitted in 2014.
	
	[3] S. Janson, T. Luczak, A. Rucinski, Random Graphs, New York, Wiley, 2000.
	
	[4] J.H. Spencer, The Strange Logic of Random Graphs, Springer Verlag, 2001.
	
	[5] M.E. Zhukovskii, A.M. Raigorodskii, Random graphs: models and asymptotic charac-
	teristics, Russian Mathematical Surveys, 70(1): 33–81, 2015.
\end{document}