\documentclass{article}
\usepackage[utf8]{inputenc}
\usepackage[T2A]{fontenc}
\usepackage{textcomp}
\usepackage{amssymb}
\usepackage{amsmath}

\title{О монадическом законе нуля и единицы для графа $G(n, p)$}
\author{Анастасия Ремизова}
\date{15 декабря 2016 г.}

\begin{document}

\maketitle

\section{Определения}

Рассмотрим {\bf модель случайных графов Эрдеша-Реньи}. Пусть $\mathfrak{G}$ --- множество, содержащее все конечные неориентированные графы $G = \langle |G|, E\rangle$. Определим функцию $p= p(n)$, $p: \omega \mapsto [0, 1]$. Определим вероятностную модель  $\mathfrak{G}(n, p) = \langle \mathfrak{G}, {\sf P}_p \rangle$: для графа $G$ c числом ребер $\epsilon$ определим ${\sf P}(\mathfrak{G}(n, p) = G) = p^{\epsilon}(1-p)^{{n \choose 2} - \epsilon}$.

Рассмотрим формулы {\bf логики первого порядка} и {\bf монадической логики второго порядка} над случайными графами: сигнатура $\sigma$ содержит символы равенства $=$ и смежности $\sim$. Формулы логики первого порядка построены из атомарных переменных $x$, $y$, $x_1$, ... и двух предикатных символов, упомянутых выше, логических связок $\neg$, $\land$, $\lor$, $\rightarrow$ и кванторов $\forall$, $\exists$ по переменным. Формулы монадической логики (второго порядка), помимо этого, могут содержать переменные, обозначающие множества, в форме $x \in X$, и кванторы по ним.

Будем рассматривать только конечные формулы. Назовем {\bf кванторной глубиной $q(\varphi)$ формулы $\varphi$} число вложенных кванторов в самой длинной цепи вложенных кванторов в этой формуле.

Нас интересуют асимптотические свойства вероятности выполнения свойства $\varphi$ для случайного графа $\mathfrak{G}(n, p)$ --- обозначим эту вероятность ${\sf P}_{n, p}(\varphi)$, а особенно существование предела этой вероятности при $n \rightarrow \infty$, который мы назовем {\bf асимптотической вероятностью $\varphi$} --- ${\sf P}_{p} (\varphi) = lim_{n \rightarrow \infty} {\sf P}_{n, p} (\varphi)$. Если этот предел существует для любой формулы логики $L$, то мы говорим, что выполняется {\bf закон сходимости} для $L$ и $p$. Если, к тому же, для каждой формулы асимптотическая вероятность равна $0$ или $1$, мы говорим, что выполняется {\bf закон $0$ и $1$}. Если асимптотическая вероятность равна $1$, мы говорим, что $\varphi$ выполняется {\bf асимптотически почти наверное (а.п.н.)}. В случае, когда для всех формул логики $L$, имеющих глубину не более чем $k$, асимптотическая вероятность равна $0$ или $1$, мы говорим, что выполняется {\bf $k$-закон $0$ и $1$}. Разумеется, закон $0$ и $1$ (для логики $L$) выполняется тогда и только тогда, когда для любого натурального $k$ выполняется $k$-закон $0$ и $1$.

В частности, если $L$ --- логика первого порядка, то мы называем эти законы {\bf FO-законами} (из-за того, что в английском языке логика первого порядка называется first order logic), а если $L$ --- монадическая логика второго порядка --- то {\bf MSO-законами} (monadic second order logic).

\section{Известные результаты}

{\bf Теорема 1 (Глебский, Коган, Лиогонький, Таланов, Фагин [1]).} При $p = const$ FO-закон $0$ и $1$ выполнен.

{\bf Теорема 2 (Кауфман, Шелах [3], Тишкиевич [4]).} При $p = const$ граф $G(n, p)$ не подчиняется MSO-закону сходимости.

Надо заметить, что доказательство ниже было приведено Тишкиевичем. Оно имеет преимущество перед более ранним доказательством Кауфмана и Шелаха: его легче перенести на случай $p = n^{-\alpha}$, $\alpha\in(0,1]$, для которого закон сходимости, как доказал Тишкиевич в упомянутой работе, также не выполнен.

Наша задача состоит в нахождении минимальной кванторной глубины, при которой MSO-закон $0$ и $1$ не выполнен для некоторого $p$, равного константе. Для получения верхней оценки мы постараемся найти кванторную глубину формулы, которая не имеет асимптотической вероятности и описана в доказательстве ниже. Однако она не приведена явно. В следующем разделе мы осуществим явную конструкцию этой формулы, чтобы оценить ее кванторную глубину.

{\bf Идея доказательства:} Рассмотрим пару графов $(G, H), G \subseteq H; |G| = \{ 0, ..., k-1\}, |H| \backslash |G| = \{k, ..., l-1\}$

Обозначим следующую формулу, назовем ее \textit{аксиомой расширения}, $Ext(G,H)$:
$$\forall x_0, ..., x_{k-1} ([\{x_0, ..., x_{k-1}\} \simeq G]  \to (\exists x_k, ..., x_{l-1} [\{x_0, ..., x_{l-1}\} \simeq H],$$

которая выражает свойство, заключающееся в том, что каждый граф, изоморфный $G$, можно дополнить до графа, изоморфного $H$. При этом сохраняется порядок вершин: изоморфизм, переводящий $x_0$, ..., $x_{l-1}$ в $H$, переводит $x_0$, ..., $x_{k-1}$ в $G$.

Фагин [2] доказал, что для произвольных двух графов $G \subseteq H$ выполнено равенство: ${\sf P}_{p}(Ext(G, H)) = 1$, в частности, ${\sf P}_{p}(Ext(\varnothing, H)) =1$. При этом ${\sf P}_{n, p}(Ext(G,H))$ равна $1$ при $n < |G|$ и ${\sf P}_{p, n}(Ext(G, H)) = 1 - {\sf P}_{n, p}(Ext(\varnothing, G))$ при $ n < |H|$.

Можно выбрать последовательность $(G_i, H_i)$ такую, что с ростом $i$ мощности $|G_i|$ и $|H_i|$ бы очень быстро возрастали: $|H_i|$ был бы много больше $|G_i|$ и $|H_{i-1}|$, и при достижении $n$ значения $|H_i|$ ${\sf P}_{p, n} (\bigwedge_{j < i} (Ext(G_j, H_j))) \approx 1$. Тогда формула, выражающая $\bigwedge_{i \in \omega} Ext(G_i, H_i)$, не имела бы асимптотической вероятности, так как:
$${\sf P}_{p, n}(\bigwedge_{i \in \omega} Ext(G_i, H_i)) \approx \min_{i \in \omega} {\sf P}_{p, n}(Ext(G_i, H_i)).$$

Зададим такую последовательность с помощью монадической формулы.

{\bf Доказательство:} Пусть $M$ --- детерминированная одноленточная машина Тьюринга, которая на вход принимает число в его унарном разложении и всегда останавливается.

Определим функцию из $\omega$ в $\omega$:
$$size_M (m) = m + space_M (m) \cdot time_M (m),$$
где $time_M (m)$ --- число шагов вычисления машиной $M$ при входе $m$, $space_M (m)$ --- число клеток, в которых побывала головка до завершения работы машины $M$.

Построим монадическую формулу $\varphi(X)$ такую, что если верно, что $G \models \varphi(X)$, то мощность $X$ равна $size_M(m)$ при некотором $m$. Мы еще вернемся к этой формуле позже.

Возьмем $\varphi(X)$ таким образом:

$$\exists L, U \exists EC, OC, ER, OR \tilde{\varphi} (L, U, EC, OC, ER, OR),$$

где $\tilde{\varphi}$ --- соединение следующих условий:
\begin{enumerate}
\item $L, U \subseteq X$
\item $EC, OC, ER, OR \subseteq U$
\item $L \cap U = \varnothing, L \cup U = X$
\item $EC \cap OC = \varnothing, EC \cup OC = U$
\item $ER \cap OR = \varnothing, ER \cup OR = U$
\item $\langle U, E \rangle$ --- прямоугольная решетка такая, что $EC$ и $OC$ ($ER$ и $OR$, соответственно) --- объединения несвязанных цепей, которые являются четными и нечетными столбцами (строками) этой решетки.
\item $E$ --- биекция из $L$ на первые $|L|$ элементов первой строки решетки.
\item Других ребер не существует, кроме, возможно, ребер между вершинами $L$.
\end{enumerate}

Мы назовем граф $X$ {\bf решеточным расширением}, если для него выполняются предыдущие 8 условий, и говорим, что он --- {\bf решеточное расширение} $L'$, если в качестве $L$ можно взять $L'$.

Сделаем так, чтобы получившийся граф $X$ соответствал вычислению машины Тьюринга $M$ на входе $1...1$ длины $|L|$. Поясним, как это будет реализовано. Считаем, что имеем алфавит $A = \{ a_1, ..., a_m \}$ и множество состояний $Q = \{q_1, ..., q_k\}$. Выберем подмножества вершин графа $X$ $X(q_i, a_j)$ --- это вершины, которые будут соответствовать пребыванию головки в состоянии $q_i$ в ячейке, которая содержит $a_j$. $Y = \bigcup_{i \in \{1, ..., k\}, j \in \{1, ..., m\} } X(q_i, a_j)$. Каждая строка решетки $U$ будет соответствовать некоторому моменту времени, таким образом, в каждой строке должна быть ровно одна вершина из $Y$. В первой строке должна быть вершина из $X(q_1, a_j)$, где $q_1$ --- начальное состояние. Каждый столбец решетки $U$ будет соответствовать некоторой ячейке ленты. Переходы вида $<q_i, a_j> \rightarrow <q_{i'}, a_{j'}, d>$ будут соответствовать вершинам из двух соседних строк: вершина из верхней строки будет лежать в $X(q_i, a_j)$, а из нижней --- $X(q_{i'}, a_{j''})$, где $a_{j''}$ --- ранее записанная буква в ячейку, соответствующую этой строке (правильная последовательность записи букв тоже должна будет поддерживаться некоторым способом). Вершина из нижней строки будет находиться на один столбец левее, правее или в том же столбце, если $d = L, R$ или $N$ соответственно. При этом во всех столбцах должна быть хотя бы одна вершина из $Y$. Тогда ширина решетки $U$ --- число столбцов ---  будет равна $space_M (|L|)$, высота решетки --- число строк --- будет равна $time_M (|L|)$ (на самом деле, $time_M (|L|)+1$, но это можно исправить тем, что мы не переходим на уровень ниже, если происходит переход в завершающее состояние). Чтобы получился граф, удовлетворяющий данным условиям, добавим к $\tilde{\varphi}$ конъюнкцию с $\exists X(q_1, a_1)$ $... $ $\exists X(q_1, a_m)$ $...$ $\exists X(q_k, a_1)$ $...$ $\exists X(q_k, a_m)$ $\psi (L, U, EC, OC, ER, OR, X(q_1, a_1),$ $...,$ $X(q_1, a_m),$ $...,$ $X(q_k, a_1),$ $...$ $X(q_k, a_m)) $. $\psi$ должна быть записана так, чтобы выражать предыдущие условия.

Для другой детерминированной одноленточной машины Тьюринга $N$ мы запишем формулу $\gamma (X, Y)$ такую, что если $G \models \gamma (X, Y)$, то $|Y| = size_{N} (|X|)$. Ее можно построить аналогично $\varphi$, где вместо множества $L$ мы будем строить расширение множества $X$, но уже не ставя по нему кванторов.

Теперь определим функцию $g: \omega \rightarrow \omega$:

$g(m) = 1 + ($ наименьшее $n > g(m-1)$ такое, что $\mu_n(\bigwedge Ext(G, H)) \geq 1 - 1/m$; конъюнкция проходит по всем решеточным расширениям $(G, H)$ таким, что $|H| \leq m)$

Очевидно, что $g$ является рекурсивной строго возрастающей функцией.

Пусть машина Тьюринга $N$ принимает унарное разложение числа $m$, выводит строку из единиц и вычисляет некоторую функцию $h > g$ такую, что она удовлетворяет условию: $h(m) = space_N (m)$.

Пусть машина Тьюринга $M$ принимает унарное разложение числа $m$ и выводит строку из единиц. Более того, $M$ вычисляет всюду определенную на множестве унарных последовательностей функцию $f$ такую, что для $m> 0$:

$$f(m) > g(size_N(size_M(m-1)))$$.

Окончательно, искомой формулой будет:
$$Ext \equiv \forall X (\varphi (X) \rightarrow \gamma (X, Y)) $$

Заметим, что $Ext$ эквивалентна $\bigwedge_{m \in \omega} Ext (G_m, H_m)$, где $G_m$ --- решеточное расширение $\varnothing$ размера $size_M (m)$, $H_m$ --- решеточное расширение $G_m$ размера $size_N (size_M(m))$.

{\bf Утверждение:} Не существует ${\sf P}_{p}(Ext)$.

Пусть $n = g(size_M (m)) - 1$ для некоторого $m$. Тогда, по построению $g$,
$${\sf P}_{n, p} Ext(\varnothing, H)) \geq 1 - 1/m$$
для всех решеточных расширений $(\varnothing, H)$ с $|H| \leq size_M(m)$. Поэтому вероятность выбора графа $X$, удовлетворяющего $\phi(X)$, с мощностью $|X|$, не меньше $1-1/m$. Но тогда $Y$, удовлетворяющего $\gamma (X, Y)$, не существует, так как тогда $|Y \backslash X| \geq g(size_M (m)) > n$, а это не так. Поэтому $\mu_n (Ext) \leq 1/m$.

Значит, нижний предел ${\sf P}_{n, p}(Ext(G_m, H_m))$ равен 0.

Теперь, пусть $n = g(size_N(size_M(m-1)))$ для некоторого $m$. Так как $size_M(m) > n$, все $X$, которые могут удовлетворять $\phi(X)$, должны иметь мощности: $size_M(0)$, ..., $size_M(m-1)$. Но ${\sf P}_{n, p}(\bigwedge Ext (G, H) ) \geq 1 - 1/m$, где конъюнкция происходит по всем $(G, H)$, $H$ --- расширение $G$, $|H| \leq size_N(size_M(m-1))$, следовательно, для каждого $X$, удовлетворяющего $\varphi(X)$, есть $Y$, удовлетворяющий $\gamma(X,Y)$, с вероятностью не меньшей $1-1/m$. А значит, ${\sf P}_{n, p}(Ext) \geq (1- 1/m)$.

Значит, верхний предел ${\sf P}_{n, p} (Ext(G_m, H_m))$ равен 1.

А потому MSO-закон сходимости не выполняется.

\section{Полученные результаты}

Мы желаем найти минимальную кванторную глубину монадической формулы, при которой MSO-закон $0$ и $1$ не выполняется.

Из доказательства утверждения следует идея для верхней оценки глубины монадической формулы, при которой для $p = const$ не выполняется закон $0$ и $1$ --- собственно, оценить глубину формулы $Ext$. Для этого построим ее явно.

Сначала построим формулу, которая показывает, что граф удовлетворяет свойствам 1)-8).

1) $L, U \subseteq X$:

$\forall x (x \in L \rightarrow x \in X)$  --- глубина 1.

Аналогично для $U \subseteq X$ и 2).

2) $EC, OC, ER, OR \subseteq U$

3) $L \cap U = \emptyset, L \cup U = X$

$\forall x \neg (x \in L \land x \in U) \land (x \in X \rightarrow x \in L \lor x \in U)$  --- глубина 1.

Аналогично для 4), 5).

4) $EC \cap OC = \emptyset, EC \cup OC = U$

5) $ER \cap OR = \emptyset, ER \cup OR = U$

6) $\langle U, E \rangle$ --- решетка. $EC, OC (ER, OR)$ --- объединения разъединённых цепей, представляющих собой чётные и нечётные столбцы (строки) решётки.

Обозначим число соседей вершины $v$ из множества $V$ как $deg (V, v)$:

$deg (v, V) = 0$ --- глубины 1:

$\forall v_1 (v_1 \in V \rightarrow (v \nsim v_1))$

$deg (v, V) = 1$ --- глубины 2:

$\exists v_1 (v_1 \in V \land v \sim v_1 \land \forall v_2 ((v_2 \in V \land v \sim v_2) \rightarrow
v_1 = v_2)) $

$deg (v, V) = 2$ --- глубины 3:

$\exists v_1 \exists v_2 (v_1 \in V \land v_2 \in V \land v_1 \neq v_2 \land v \sim v_1 \land v
\sim v_2 \land \forall v_3((v_3 \in V \land v \sim v_3) \rightarrow (v_3 = v_1 \lor v_3 = v_2)))$

Множество $V$ и ребра между его вершинами представляют собой объединение разъединённых цепей; обозначим формулу $UnionChains(V)$ --- она глубины 4.

$\forall v (v \in V \rightarrow (deg (v, V) = 1 \lor deg(v, V) = 2)) \land $

$\forall V_1 (V_1 \subseteq V \rightarrow \exists v_1 (v_1 \in V_1 \land (deg (v_1, V_1) = 0 \lor deg (v_1, V_1) = 1)))$

Вторая часть формулы отвечает за отсутствие циклов в множестве.

Нужно просто подставить $EC$ ($ER$, $OC$,$OR$) в формулу $UnionChains(V)$, чтобы получить,
что они - объединения разъединённых цепей.

Формула, показывающая, что $V$ является множеством, состоящим из одной цепи, при условии, что $V$ --- подмножество вершин объединения разъединённых цепей, --- $ChainInUnion(V)$ --- глубины 4:

$\land \exists BE (\exists v_1 \exists v_2 (v_1 \in V \land v_2 \in V \land v_1 \in BE \land v_2 \in BE \land
\forall v (v \in BE \rightarrow v = v_1 \land v = v_2)) \land$

$\forall v ((v \in BE \rightarrow deg(v, BE) = 1) \land ((v \in V \land v \notin BE) \rightarrow
\neg(deg(v,V) = 1 \lor deg(v, V) = 0))$

$BE$ --- множество, состоящее из начала и конца цепи.

Формула, показывающая, что можно построить инъекцию из $V_1$ в $V_2$, которая
отображает вершину из $V_1$ в ее единственного соседа из $V_2$ --- $Injection(V_1, V_2)$ --- глубины 3:

$\forall v_1 (v_1 \in V_1 \rightarrow \exists v_2 (v_2 \in V_2 \land v_1 \sim v_2 \land
\forall v_3 ((v_3 \in V_2 \land v_1 \sim v_3) \rightarrow (v_3 = v_2))))$

Тогда можно определить биекцию из $V_1$ в $V_2$ --- $Bijection(V_1, V_2)$ --- глубины 3:

$Injection(V_1, V_2) \land Injection(V_2, V_1)$

Дополнительное условие для цепей --- чтобы биекция имела правильный вид и соединяла соседей в цепи с соседями в другой цепи --- $CorrectConnection(V_1, V_2)$ --- глубины 4:

$\forall v_1 \forall v_2 \forall v_3 \forall v_4 ((v_1 \in V_1 \land v_2 \in V_2 \land v_3 \in V_1
\land v_4 \in V_2 \land v_1 \sim v_2 \land v_1 \sim v_3 \land v_3 \sim v_4) \rightarrow v_2 \sim v_4)$

Формула, означающая, что цепи соединены правильным образом --- $Connected(V_1, V_2)$ --- глубины 4:

$Bijection(V_1, V_2) \land CorrectConnection(V_1, V_2)$

Формула, показывающая, что любые 2 цепи из подмножеств цепей в четных и нечетных
столбцах либо соединены, либо не имеют общих ребер --- $AllChainConnection(EC, OC)$ --- глубины 6.

$\forall C_1 \forall C_2 ((((C_1 \subseteq EC \land C_2 \subseteq OC)
\lor (C_1 \subseteq OC \land C_2 \subseteq EC)) \land ChainInUnion(C_1) \land ChainInUnion(C_2)) \rightarrow (Connected(C_1, C_2) \lor (\forall v (v \in C_1 \rightarrow deg(v, C_2) = 0)))))$

Формула, показывающая, что $C$ --- крайняя цепь из $EC$ --- $LastChain(C, EC, OC)$ ---
глубины 4:

$ChainInUnion(C) \land C \subseteq EC \land \forall v ((v \in C) \rightarrow deg(v, OC) = 1)$

Формула, показывающая, что любая цепь из $EC$, не совпадающая с $C_1$ и $C_2$, соединена с 2 другими из $OC$ --- $OtherChains(EC, OC, C_1, C_2)$ --- глубины 4:

$\forall v ((EC(v) \land v \notin C_1 \land v \notin C_2) \rightarrow deg(v, OC) = 2))$

Формула, определяющая решетку --- $Grid(EC, OC)$ --- глубины 6:

$AllChainConnection(EC, OC) \land \exists C_1 \exists C_2 ((LastChain(C_1, EC, OC) \lor
LastChain(C_1, OC, EC)) \land (LastChain(C_2, EC, OC) \lor LastChain(C_2, OC, EC)) \land
OtherChains(EC, OC, C_1, C_2) \land OtherChains(OC, EC, C_1, C_2))$

Аналогично можно записать $Grid(ER, OR)$.

7. $E$ --- биекция из $L$ на первые $|L|$ элементов первой строки.

$\exists R_1 (LastChain(C_1, ER, OR) \land Injection(L, C_1) \land
CorrectConnection(L, C_1) \land \exists v (deg (v, C_1) = 1 \land \exists v_1( v_1 \in L \land v_1 \sim v))$

8. Не существует иных ребер, кроме, возможно, между вершинами $L$ ---
следует уже из 7.
 
Дальнейшая работа состоит в том, чтобы построить формулу, определяющую соответствие данного графа некоторой конкретной машине Тьюринга.

\section{Список литературы}

\quad \enspace  [1] Ю.В. Глебский, Д.И. Коган, М.И. Лиогонький, В.А.Таланов, Объем и доля выполнимости формул узкого исчисления предикатов, \textit{Кибернетика}, 1969, \textbf{2}: 17-26.

[2] R. Fagin, Probabilities in finite models, \textit{J. Symbolic Logic}, 1976, \textbf{41}: 50-58.

[3] M. Kaufmann, S. Shelah, On random models of finite power and monadic logic, \textit{Discrete Mathematics}, 1985, \textbf{54(3)}: 285-293.

[4] J. Tyszkiewicz, On Asymptotic Probabilities of Monadic Second Order Properties, \textit{Lecture Notes in Computer Science}, 1993, \textbf{702}: 425-439.
\end{document}